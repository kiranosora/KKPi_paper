
%\section{Introduction}
%\label{sec:introduction}
%
%
%
%%------------------------------------------------------------------------------
%\section{BESIII Detector and Data Sets}
%\label{sec:detector_dataset}
%The BESIII detector is a magnetic spectrometer~\cite{BESIII} located at the Beijing Electron Positron Collider (BEPCII)~\cite{BEPCII}.
%The subdetectors surrounded by the superconducting solenoidal magnet, which provide a 1.0 T magnetic field, starting from the interaction point consist of  a main drift chamber (MDC), a plastic scintillator time of flight counters (TOF), a CsI(Tl) electromagnetic calorimeter (EMC).
%Charged particle identification is performed by combining the ionization energy loss  measured by MDC and the time-of-flight measured by TOF.
%EMC provides the shower information to reconstruct photons.
%Outside the solenoidal magnet is a multi-gap resistive-plate chamber system (MRPC), which provides muon identification.
%
%Monte Carlo (MC) samples are produced with the GEANT4-based~\cite{GEANT4} package.
%An inclusive MC sample is produced at $E_{cm}$ = 4.178 GeV.
%The sample includes all known open charm decays, the continuum processes ( $e^{+}e^{-} \rightarrow q\bar{q}$, q = u, d and s), Bhabha scattering, $\mu^{+}\mu^{-}$, $\tau^{+}\tau^{-}$, diphoton process and the $c\bar{c}$ resonances $J/\psi$, $\psi(3686)$ and $\psi(3770)$ via the initial state radiation (ISR).
%The generator CONEXC~\cite{CONEXC} is used to model the open charm processes directly produced via $e^{+}e^{-}$ annihilation.
%The simulation of ISR production of $\psi(3770)$, $\psi(3686)$ and $J/\psi$ is performed with the KKMC~\cite{KKMC} package.
%The final-state radiation from charged tracks is produced by the PHOTOS package~\cite{PHOTOS}.
%The known decays with BFs taken from Particle Data Group (PDG)~\cite{PDG} are simulated with the EVTGEN package~\cite{EVTGEN} and
%the unknown decays are generated with the LUNDCHARM model~\cite{LUNDCHARM}.
%The inclusive MC (generic MC) is used to perform background analysis. 
%%------------------------------------------------------------------------------
%\section{Event Selection}
%\label{chap:event_selection}
%%\label{selection}
%For charged tracks except for those from $K_{S}^{0}$ decays, the polar angles ($\theta$) respect to the beam axis  must satisfy $|cos\theta| < 0.93$.
%The distances of charged tracks from the interaction point in the transverse plane and along the beam direction should be less than 1 cm and 10 cm, respectively.
%
%Photons are reconstructed from the cluster shower in EMC.
%The deposit energy of the photons from the endcap ($0.86 < |cos\theta| < 0.92$) should be larger than 50 MeV and that of the photons from the barrel ($|cos\theta| < 0.8$) should be larger than 25 MeV.
%Furthermore, the shower time from the event start time should be within 700 ns.
%
%We reconstruct $\pi^{0} (\eta)$ candidates through $\pi^{0} \rightarrow \gamma\gamma$ ($\eta \rightarrow \gamma\gamma$).
%The diphoton invariant mass $M_{\gamma\gamma}$ for $\pi^{0}$ and $\eta$ should be in the range of 0.115 $< M_{\gamma\gamma} <$ 0.150 GeV/$c^{2}$ and 0.490 $< M_{\gamma\gamma} <$ 0.580 GeV/$c^{2}$, respectively.
%Then we perform a fit to constrain $M_{\gamma\gamma}$ to the $\pi^{0}$ or $\eta$ nominal mass ~\cite{PDG}.
%The $\pi^{0}$ or $\eta$ candidates with a $\chi^{2}_{1C}$ less than 30 are retained.
%
%Kaons and pions are identified by the combining the information of dE/dx in the MDC and the time-of-flight from the TOF.
%If the probability of the kaon hypothesis is larger than that of the pion hypothesis, the charged track is identified as a kaon.
%Otherwise, the track is identified as a pion.
%
%We vote the $\pi^{\pm}$ and $pi^{0}$ whose momentum is less than 0.1 GeV to remove soft $\pi^{\pm}$ and $\pi^{0}$ from $D^{*}$ decays.
%
%The $\eta^{'}$ candidates are reconstructed via the process $\eta^{'} \rightarrow \pi^{+}\pi^{-}\eta$.
%The candidates with an invarian mass for $\pi^{+}\pi^{-}\eta$ falling into the range of $[0.938, 0.978]$ GeV/$c^{2}$ are retained.
%
%$\pi^{+}\pi^{-}$ pairs are used to reconstruct $K_{S}^{0}$ mesons.
%The polar angles $\theta$ of the two pions should satisfy $|cos\theta| < 0.93$.
%The distances of the two pions should from the interaction point along the beam direction should be less than 20 cm. 
%The invariant mass $M(\pi^{+}\pi^{-}$ of $\pi^{+}\pi^{-}$ pairs  should satisfy $0.487 < |cos\theta| < 0.511$ GeV/$c^{2}$.
%The decay length and the decay length error of $K_{S}^{0}$ are obtained with second vertex fit.
%We require the decay length over the decay length error to be less than 2.
%
%Then the $D_{s}$ candidates are reconstructed from $K^{\pm}$, $\pi^{\pm}$, $\eta$, $\eta^{'}$, $K_{S}^{0}$ and $\pi^{0}$.
%We reserve the candidates with an invariant mass $M(D_{s}$ falling in the mass window $[1.87, 2.06]$ GeV/$c^{2}$ and an recoiling mass $M_{rec}$ falling in the mass window $[2.051, 2.180]$ GeV/$c^{2}$.
%$M_{rec}$ is defined as:
%\begin{equation}
%    M_{rec} = \sqrt{(E_{cm} - \sqrt{| \vec p_{D_{s}} |^{2} + m_{D_{s}}^{2}})^{2} - |\vec p_{cm} - \vec p_{D_{s}} | ^{2}} \; , \label{con:inventoryflow}
%\end{equation}
%where $p_{D_{s}}$ is the momentum of $D_{s}$ candidate, $m_{D_{s}}$ is $D_{s}$ mass quoted from PDG~\cite{PDG},
%and $\vec p_{cm}$ and $\vec p_{D_{s}}$ are the momentum of the initial state and the decay products of the $D_{s}$ candidate, respectively. 
%


\section{Partial Wave Analysis in the Low $K^{+}K^{-}$ Mass Region}
\label{Amplitude Analysis}
As it's very hard to distinguish $a_{0}(980)$ and $f_{0}(980)$ in the low $K^{+}K^{-}$ mass region, we use S(980) to denote the $a_{0}(980)$ and $f_{0}(980)$ resonances and we perform a model-independent partial wave analysis (MIPWA) to extract S(980) line shape near the threshold of $K^{+}K^{-}$ mass spectrum.

After the selection in Sec.~\ref{chap:event_selection}, for each $D_{s}$ candidates decaying to  $K^{+}K^{-}\pi^{+}$ in an event, all daughter tracks are added to apply a 1C kinematic fit constraining the mass of $D_{s}$.
Then we select the candidate with minimum $\chi^{2}_{1C}$ as the best candidate in an event.
We define the invariant mass of the $D_{s}$ with opposite charge to the signal $D_{s}$ as $M_{oth} = \sqrt{ (E_{cm} - \sqrt{| \vec p_{D_{s}} |^{2} + m_{D_{s}}} - E_{\gamma})^{2} - |\vec p_{cm} - \vec p_{D_{s}} - \vec p_{\gamma}| ^{2}   }$, where $E_{\gamma}$ and $\vec p_{\gamma}$ refer to the energy and momentum of gamma from the process $D_{s}^{*} \rightarrow D_{s}\gamma$.
The good gamma with $M_{oth}$ closest to the nominal $D_{s}$  mass~\cite{PDG} is taken as the gamma from the decay $D_{s}^{*} \rightarrow D_{s}\gamma$.

Multi-variable analysis (MVA) method~\cite{TMVA} is used to suppress the background.
With the gradient boosted decision tree (BDTG) classifier provided by MVA, we train MVA separately with two sets of variables for the two categories depending on the $D_{s}^{+}$ origin.
Two categories of events are selected in a $M_{rec} - \Delta M$ 2D plane, where $\Delta M \equiv M(D_{s}^{+}\gamma) - M(D_{s}^{+})$, $M(D_{s}^{+})$ is the invariant mass of signal $D_{s}$ and $M(D_{s}^{+}\gamma)$ refers to the invariant mass of $D_{s}$ and the gamma from $D_{s}^{*} \rightarrow D_{s}\gamma$, as shown in Fig.~\ref{2DAll}.

\begin{figure*}[htbp]
 \centering
 \mbox{
  %\vskip -1.5cm
  \begin{overpic}[width=0.48\textwidth]{plot/2DAll.png}
  \end{overpic}
 }
 \caption{Two dimensional plane of ${M_{rec}}$ versus ${\Delta{M} \equiv M(D_{s}^{+}\gamma) - M(D_{s}^{+})}$ from the simulated ${D_{s}^{+} \rightarrow K^{+}K^{-}\pi^{+}}$ decays. The red (green) dashed lines mark the mass window for the $D_{s}^{+}$ Cat.~\#0 ( Cat.~\#1) around the $M_{rec}$ ($\Delta{M}$) peak. }
\label{2DAll}
\end{figure*}



%
%
%%------------------------------------------------------------------------------
%\section{Amplitude Analysis}
%\label{Amplitude Analysis}
%
%
%
%
%
%
%%------------------------------------------------------------------------------
%\section{Branching Fraction}
%
%
%
%%------------------------------------------------------------------------------
%
%
%%------------------------------------------------------------------------------
%\section{Conclusion}
%\label{CONLUSION}
%
%
%
%
%\begin{acknowledgements}
%\label{sec:acknowledgement}
%\vspace{-0.4cm}
%\end{acknowledgements}
%
